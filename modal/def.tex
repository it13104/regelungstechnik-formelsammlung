Partialbruchzerlegung der Übertragungsfunktion $\Rightarrow$ Modus
\\

\begin{enumerate}
    \item Alle Pole sind \underline{reell} und \underline{einfach}:
    \begin{align*}
    G(s) = \frac{b_1}{s-p_1}+ \cdots +\frac{b_n}{s-p_n}
    \end{align*}
    \begin{align*}
    \dot{\bs{x}} = 
    \begin{bmatrix}[1]
    p_1 & 0 &\cdots & 0 \\
    0&p_2 &\cdots & 0  \\
    \vdots&\vdots&\ddots&\vdots\\
    0&0&\cdots&p_n
    \end{bmatrix}\bs{x}+ 
    \begin{bmatrix}
    1\\ 1 \\ \vdots \\1
    \end{bmatrix} u     
    \end{align*}
    \begin{align*}
    y= \begin{bmatrix}
    b_1 & b_2&\cdots &b_n
    \end{bmatrix} \bs{x}
    \end{align*}
    \textbf{Jordanform -  wenn die Nullstellen nicht reell sind:}
    \item Komplexes Polpaar $\Rightarrow$ Partialbruch der Art.
    Blockmatrizen für das Polpaar werden an entsprechender Stelle eingefügt
    \begin{align*}
        \frac{b_{i+1}s+b_i}{s^2+a_is+a_{i+1}}
    \end{align*}
    \begin{align*}
        \dot{\bs{x}} = \begin{bmatrix}
            0 & 1\\
            -a_{i+1} & -a_i
        \end{bmatrix}\bs{x} + \begin{bmatrix}
        0\\1
        \end{bmatrix}u & & \bs{y} = \begin{bmatrix}
        b_i & b_{i+1}
        \end{bmatrix}\bs{x}
    \end{align*}
    
    \item Ein $m$-facher Pol $\frac{b}{(s-p)^m}$
    \begin{align*}
        \dot{\bs{x}} = \begin{bmatrix}
            p&1&\cdots&0&0\\
            0&p&\cdots&0&0\\
            \vdots&\vdots&\ddots&\vdots&\vdots\\
            0&0&\cdots&p&1\\
            0&0&\cdots&0&p
        \end{bmatrix}\bs{x} + \begin{bmatrix}
        0\\0\\\vdots\\0\\1
        \end{bmatrix} u &&\bs{y} = \begin{bmatrix}
        b&0&\cdots&0&0
        \end{bmatrix}\bs{x}
    \end{align*}
    
\end{enumerate}
