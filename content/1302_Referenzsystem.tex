\begin{tcolorbox}[colback=white!10!white,colframe=green!30!black,title=Referenzsystem] 
Das ist, dass im eingeschwungenen Zustand der Ausgang $y = r$ ist. Mit $u = -\bs{Kx}+r$ als Ansatz.
\tcblower
\textbf{Vorgehensweise}
\begin{enumerate}
    \item Gleichung \begin{align*}
        \begin{bmatrix}
        \bs{F} &\bs{G}\\ \bs{H} & J
        \end{bmatrix}\begin{bmatrix}
        \bs{N_x}\\N_u
        \end{bmatrix} = \begin{bmatrix}
            0\\1
        \end{bmatrix} && \begin{bmatrix}
        \bs{N_x}\\N_u
        \end{bmatrix} = 
        \begin{bmatrix}
        \bs{F} &\bs{G}\\ \bs{H} & J
        \end{bmatrix}^{-1} \begin{bmatrix}
        0\\1
        \end{bmatrix}
    \end{align*}
    \item Damit ergibt sich \begin{align*}
    u &= N_u  *r-\bs{K}(\bs{x}-\bs{N_x}r)\\
    &= -\bs{Kx}+\underbrace{(N_u+\bs{KN_x})}_{\bar{N}}r
    \end{align*}
\end{enumerate}
\end{tcolorbox}

