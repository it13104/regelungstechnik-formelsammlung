\begin{tcolorbox}[colback=white!10!white,colframe=green!30!black,title=Stabilität] 
Alle Pole der Übertragungsfunktion müssen in der LHE sein. \textbf{Steuerung} stabilisiert nicht. 
\textbf{Regelung:}

    \begin{figure}[H]
        \centering
        \includegraphics[width=.7\textwidth]{content/img/regler}
\end{figure}
    \begin{align*}
    &G(s) = \frac{b(s)}{a(s)}  && D_{OL} = \frac{c(s)}{d(s)}
    &1+ G D_{CL} = 0\\ 
    &1 + \frac{bc}{ad} = 0  &&
    a(s)d(s) + b(s)c(s) = 0 
    \end{align*}
Die Regelung reduziert die Auswirkung von Störung um $1+ AK$

\end{tcolorbox}

\begin{tcolorbox}[colback=white!10!white,colframe=green!30!black,title=Sensitivität] 
    Sensitivität beschreibt die Reaktion des Systems auf Änderung im Parameter
    \begin{align*}
        & S_{A}^{T_{CL}} = \frac{A}{T_{CL}}\frac{d T_{CL}}{dA}
        & |S_{G}^{T_{CL}}| = \frac{1}{1+G(i\omega_0)D(i\omega_0)}
        \end{align*}
    \begin{align*}
        &T_{CL} -\text{ Closed-Loop Übertragungsfunktion} & A - \text{Parameter}
    \end{align*}
\end{tcolorbox}
%ROUTH-Criterion
\input{content/routh.tex}
%CLM-Criterion
\input{content/clm.tex}
%NYQUIST-Criterion
\input{content/nyquist.tex}

