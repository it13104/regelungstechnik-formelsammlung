\section{LQ-Regler - optimaler Regler}
\begin{tcolorbox}[colback=white!10!white,colframe=green!30!black] 
    \textbf{Anforderung an Tuning-Matrizen:}
    \begin{align*}
        \bs{P} = \bs{P}^T \succeq 0 &&     \bs{Q} = \bs{Q}^T \succeq 0  &&  \bs{R} = \bs{R}^T \succ 0
    \end{align*}
    
    \tcblower
    \begin{enumerate}
        \item Löse Gleichung     \begin{align*}
        \bs{0} = -(\bs{PF})^T - \bs{PF} -\bs{PG}\bs{R}^{-1}(\bs{PG})^T  -\bs{Q} = \bs{0}
        \end{align*}
        \item optimaler Regeleingang
        \begin{align*}
            u(t) = - \bs{Kx}(t) = -(\underbrace{\bs{R}^{-1}\bs{G}^T\bs{P}}_{K})\bs{x}(t)
        \end{align*}
        \item Eine Lösung ist gegeben, wenn alle instabilen Zustände steuerbar sind  -\textbf{stabilisierbar}
        \item Instabile Moden müssen durch die Kostenfunktion erfasst werden
        \item Stabil, wenn das System \begin{align*}
            \dot{\bs{x}} = \bs{Fx} && \bs{y} = \bs{Q}^{\frac{1}{2}}\bs{x} 
        \end{align*}
        \textbf{detektierbar} ist. Alle instabilen Zustände sind beobachtbar.
        \item Wenn $[\bs{H},\bs{F}]$ detektierbar ist typischerweise $Q = \bs{H}^T\bs{H}$ 
    \end{enumerate}

\begin{tcolorbox}[colback=white!10!white,colframe=gray!30!black] 
    Stabilisierbarkeit lässt sich durch die B-Matrix, Detektierbarkeit durch die C-Matrix erkennen, wenn das System in Modalform vorliegt.
\end{tcolorbox}    
\begin{tcolorbox}[colback=white!10!white,colframe=gray!30!black] 
    Ein Maß für die Güte eines Reglers ist die Kostenfunktion $J$. Die Funktion spiegelt die Kosten  für einen Regler wider. Man minimiert die Kostenfunktion. $\Rightarrow$ Ricatti-Gleichung
    
    \begin{align*}
        J = \int_{0}^{\infty}(\bs{x}^T(t)\bs{Qx}(t)+\bs{u}^T(t)\bs{Ru}(t))dt
    \end{align*}
\end{tcolorbox}    
    
\end{tcolorbox}
