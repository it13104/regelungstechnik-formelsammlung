\section{Pole und NST in der s-Ebene}
\begin{tcolorbox}[colback=white!10!white,colframe=green!30!black,title=PT$_2$ - Glied] 
	\begin{figure}[H]
		\begin{minipage}{.3\textwidth}
			\centering
			\includegraphics[width=1\textwidth]{content/img/winkel}
		\end{minipage}%
		\hspace{1cm}
		\begin{minipage}{.6\textwidth}
			Instabilität beginnt in der rechten Halbebene. 
			
			Für die  \textbf{Dämpfung} gilt, je kleiner der Winkel $\theta$, desto kleiner ist die Dämpfung: \begin{align*}
				\theta = \arcsin{\zeta}
			\end{align*}
		\end{minipage}%
		
	\end{figure}
	\textbf{Charakteristische Gleichung:} $s^2+2*\zeta*\omega_n*s+\omega_n^2$
\end{tcolorbox}

\begin{tcolorbox}[colback=white!10!white,colframe=green!30!black,title=Auswirkung der NST und Polenlage] 
	\begin{figure}[H]
	
			\centering
			\includegraphics[width=1\textwidth]{content/img/lageAntwort}

	\end{figure}
\tcblower
	\textbf{Polstelle (reell + komplexes Polpaar)}
	\begin{table}[H]
		\centering
		\begin{tabular}{ccc}
			\hline Re & Im & Auswirkung \\ 
			\hline $<$ 0 & = 0 & stabil - keine Schwingung \\ 
			\hline $>$ 0  & =0 & instabil - keine Schwingung \\ 
			\hline $=0$ & =0 & instabil - keine Schwingung \\ 
			\hline\hline $< 0$ & $\not =$ & stabil - Schwingung \\ 
			\hline $> 0$  &  $\not =$ & instabil - Schwingung \\ 
			\hline = 0  & $\not =$ & Dauerschwingung \\ 
			\hline 
		\end{tabular} 
	\end{table}
	
	\textbf{Nullstelle:}
	\begin{table}[H]
		\centering
		\begin{tabular}{p{2cm}p{3cm}}
			\hline $< = 0 $ & minimalphasiges System (evtl. Überschwung) \\ 
			\hline $> = 0 $ & nichtminimalphasiges System (Systemantwort erst entgegen der Sprunganregung) \\ 
			\hline $= 0 $ & $lim_{t\rightarrow\infty}y(t)\rightarrow 0$ \\ 
			\hline 
		\end{tabular} 
	\end{table}
\end{tcolorbox}

\begin{tcolorbox}[colback=white!10!white,colframe=green!30!black,title=Kenngrößen] 
	\begin{figure}[H]
		\includegraphics[width=1\linewidth]{content/img/numb}
	\end{figure}
	
\begin{minipage}{.45\textwidth}
		Steigzeit (0,1 - 0,9):
	\begin{align*}
		t_r &\approx \frac{1,8}{\omega_n} 
	\end{align*}
\end{minipage}
\begin{minipage}{.45\textwidth}
 Einschwingzeit (Abklang auf) $\pm 1\%$
 \begin{align*}
 t_s  &= \frac{4,6}{\zeta\omega_n} = \frac{4,6}{\sigma}
 \end{align*}
\end{minipage}
\begin{minipage}{.45\textwidth}
 Überschwingweite:
 \begin{align*}
 M_p = e^{\frac{-\pi \zeta}{\sqrt{1-\zeta^2}}} \\ 0 \leq \zeta \leq 1
 \end{align*}
\end{minipage}
\begin{minipage}{.45\textwidth}
 Anstiegszeit:
 \begin{align*}
 t_p &= \frac{\pi}{\omega_n \sqrt{1-\zeta^2}} =\frac{\pi}{\omega_d}
 \end{align*}

\end{minipage}	
\begin{minipage}{.45\textwidth}
	Gedämpfte Kreisfrequenz:
	\begin{align*}
		\omega_d = \omega_n*\sqrt{1-\zeta^2}
	\end{align*}
	
\end{minipage}
\begin{minipage}{.45\textwidth}
	Ungedämpft zu gedämpft
	\begin{align*}
	\omega_n^2 = \sigma^2+\omega_d^2
	\end{align*}
	
\end{minipage}
	Dämpfung $\zeta = \sqrt{\frac{\ln(M_p)^2}{\pi^2+\ln(M_p)^2}}$
	
\end{tcolorbox}
\begin{tcolorbox}[colback=white!10!white,colframe=green!30!black,title=Stabilität] 
Alle Pole der Übertragungsfunktion müssen in der LHE sein. \textbf{Steuerung} stabilisiert nicht. 
\textbf{Regelung:}

    \begin{figure}[H]
        \centering
        \includegraphics[width=.7\textwidth]{content/img/regler}
\end{figure}
    \begin{align*}
    &G(s) = \frac{b(s)}{a(s)}  && D_{OL} = \frac{c(s)}{d(s)}
    &1+ G D_{CL} = 0\\ 
    &1 + \frac{bc}{ad} = 0  &&
    a(s)d(s) + b(s)c(s) = 0 
    \end{align*}
Die Regelung reduziert die Auswirkung von Störung um $1+ AK$

\end{tcolorbox}

\begin{tcolorbox}[colback=white!10!white,colframe=green!30!black,title=Sensitivität] 
    Sensitivität beschreibt die Reaktion des Systems auf Änderung im Parameter
    \begin{align*}
        & S_{A}^{T_{CL}} = \frac{A}{T_{CL}}\frac{d T_{CL}}{dA}
        & |S_{G}^{T_{CL}}| = \frac{1}{1+G(i\omega_0)D(i\omega_0)}
        \end{align*}
    \begin{align*}
        &T_{CL} -\text{ Closed-Loop Übertragungsfunktion} & A - \text{Parameter}
    \end{align*}
\end{tcolorbox}
%ROUTH-Criterion
\input{content/routh.tex}
%CLM-Criterion
\input{content/clm.tex}
%NYQUIST-Criterion
\input{content/nyquist.tex}


