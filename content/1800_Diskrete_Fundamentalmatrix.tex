\setcounter{section}{17}
\section{Diskrete Fundamentalmatrix}
\begin{tcolorbox}[colback=white!10!white,colframe=green!30!black,title=Definition] 
    \begin{align*}
        \bs{A}_d(T)= e^{\bs{A}T} && B_d = \int_{0}^{T}\bs{A}_d
        (\tau)d\tau    \cdot \bs{B} 
    \end{align*}
    Mit Taylorzerlegung kann man $A_d$ berechnen:
    \begin{align*}
        e^{\bs{A}t}  &= \bs{I} + \bs{A}t+\bs{A}^2\frac{t^2}{2!} + \bs{A}^3\frac{t^3}{3!}+ \ldots\\
        \frac{d}{dt}e^{\bs{A}t} &= \bs{A} + \bs{A}^2t+\bs{A}^3\frac{t^2}{2!} + \ldots
    \end{align*}
\end{tcolorbox}
\begin{tcolorbox}[colback=white!10!white,colframe=blue!30!black,title=KOCHREZEPT: Fundamentalmatrix Berechnung] 
    \begin{align*}
        \bs{A}_d(T) &= \mathcal{L}^{-1}\{(s\bs{I}-\bs{A})^{-1}\}\\
        \bs{B}_d(T) &= \mathcal{L}^{-1}\{(\frac{1}{s}\bs{A}_d(s)\cdot \bs{B})\}
    \end{align*}
    Für $2\times2$-Matrix lässt sich $A_d$ so bestimmen:
    \begin{align*}
        \bs{A}_d(s) &= \frac{1}{|s\bs{I}-\bs{A}|} adj(s\bs{I}-\bs{A})\\
        \bs{A}_d(T) &= \mathcal{L}^{-1}\{A_d(s)\}
    \end{align*}
    \tcblower
    \begin{align*}
        \mathcal{C}_d &= \begin{bmatrix}
        \bs{B}_d & \bs{A}_d\bs{B}_d & \ldots & \bs{A}_d^{n-1}\bs{B}_d
        \end{bmatrix}\\
        \mathcal{O}_d &= \begin{bmatrix}
        \bs{C} \\\bs{C}\bs{A}_d \\ \vdots \\ \bs{C}\bs{A}_d^{n-1}
        \end{bmatrix}
    \end{align*}
    \begin{tcolorbox}[colback=white!10!white,
                      colframe=gray!70!black,
                      title=Steuerbarkeit und Beobachtbarkeit]
        Das zeitdiskrete System ($\bs{A}_d,\bs{B}_d$) ist vollständig 
        \textbf{steuerbar} bzw. \textbf{beobachtbar}, wenn kontinuierliches 
        System ($\bs{A},\bs{B}$) es ist und wenn die Abtastzeit folgendes erfüllt:
        \begin{align*}
            T < \frac{\pi}{\omega_{j,\text{max}}}
        \end{align*}
        $\omega_{j,\text{max}}$ ist betraglich größter Eigenwert von $A$
    \end{tcolorbox}
\end{tcolorbox}
