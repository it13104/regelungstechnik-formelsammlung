	\section{Wurzelortskurve}
\begin{tcolorbox}[colback=white!10!white,colframe=blue!50!white,title=Konstruktionsregeln]
	$n$ - Anzahl der Pole ($p_i$)\hspace{1cm}
	$m$ - Anzahl der NST ($z_i$)
	\tcblower  %makes a horizontal line
	\textbf{Regel 1:}\\
	$n-m$ Zweige gehen zu den  Asymptoten\\
	$m$ Zweige enden in den NSTs\\

	\textbf{Regel 2:}\\
		Ein Punkt ist auf WOK, wenn die Summe der Pole und NST rechts vom Punkt ungerade ist (0 ist ungerade) (Nur Pole und NST auf der reellen Achse)\\
		
	\textbf{Regel 3:}\\
	Asymptoten schneiden sich im Schwerpunkt $\alpha$ und haben Winkel $\Phi_l$, $l =1,\dots,n-m$
	\begin{align*}
		\Phi_l &= \frac{180^\circ}{n-m}+\frac{360^\circ(l-1)}{n-m}\\
		\alpha &=	\frac{\sum p_i - \sum z_i}{n-m}
	\end{align*}
		\textbf{Regel 4:}\\
		Beachte Laufindex: $l=1,\dots,q$\\
		Austrittswinkel  der Zweige $q$-Facher Pol. $\Phi_i$- Winkel aus der Sicht der Polstelle. $\Psi_i$ - Winkel aus der Sicht der Nullstelle.
		\begin{align*}
			\Phi_{l,Beginn} &= \frac{1}{q}\left (\sum\Psi_i-\sum_{i\not= l}\Phi_i-180^\circ-360^\circ(l-1)\right )
		\end{align*}
		Eintrittswinkel in die $q$-fachen Nullstellen:
		\begin{align*}
			\Psi_{l,End} &= \frac{1}{q}\left (\sum\Phi_i-\sum_{i\not= l}\Psi_i+180^\circ+360^\circ(l-1)\right )
		\end{align*}
		\textbf{Regel 5:}\\
		Austrittswinkel aus einem Verzweigungspunkt ($q$-facher Pol):
		\begin{align*}
			\Psi_l = \frac{180^\circ}{q}+360^\circ\frac{l-1}{q}
		\end{align*}
		\textbf{Regel 6:}\\
		Verzweigungspunkte Pole (Nullstellen):
		\begin{align*}
			\sum_{i=1}^{m}\frac{1}{s-z_i}-\sum_{i=1}^{n}\frac{1}{s-p_i} =0
		\end{align*}
		\textbf{Charakteristische Gleichung:}
		\begin{align*}
			1+ K*L(s) = 0
		\end{align*}
		
		\textbf{Berechnung von $\bs{K}$:}
		
		Das gesuchte $s$ in die Gleichung einsetzen und die Gleichung lösen
		\begin{align*}
			K = -\frac{\prod_{i}(s-p_i)}{\prod_{i}(s-z_i)}
		\end{align*}
\end{tcolorbox}