\begin{tcolorbox}[colback=white!10!white,colframe=blue!70!black,title=KOCHREZEPT: Modalform]
	Partialbruchzerlegung der Übertragungsfunktion $\Rightarrow$ Modus
\\

\begin{enumerate}
	\item Alle Pole sind \underline{reell} und \underline{einfach}:
	\begin{align*}
	G(s) = \frac{b_1}{s-p_1}+ \cdots +\frac{b_n}{s-p_n}
	\end{align*}
	\begin{align*}
	\dot{\bs{x}} = 
	\begin{bmatrix}[1]
	p_1 & 0 &\cdots & 0 \\
	0&p_2 &\cdots & 0  \\
	\vdots&\vdots&\ddots&\vdots\\
	0&0&\cdots&p_n
	\end{bmatrix}\bs{x}+ 
	\begin{bmatrix}
	1\\ 1 \\ \vdots \\1
	\end{bmatrix} u 	
	\end{align*}
	\begin{align*}
	y= \begin{bmatrix}
	b_1 & b_2&\cdots &b_n
	\end{bmatrix} \bs{x}
	\end{align*}
	\textbf{Jordanform -  wenn die Nullstellen nicht reell sind:}
	\item Komplexes Polpaar $\Rightarrow$ Partialbruch der Art.
	Blockmatrizen für das Polpaar werden an entsprechender Stelle eingefügt
	\begin{align*}
		\frac{b_{i+1}s+b_i}{s^2+a_is+a_{i+1}}
	\end{align*}
	\begin{align*}
		\dot{\bs{x}} = \begin{bmatrix}
			0 & 1\\
			-a_{i+1} & -a_i
		\end{bmatrix}\bs{x} + \begin{bmatrix}
		0\\1
		\end{bmatrix}u & & \bs{y} = \begin{bmatrix}
		b_i & b_{i+1}
		\end{bmatrix}\bs{x}
	\end{align*}
	
	\item Ein $m$-facher Pol $\frac{b}{(s-p)^m}$
	\begin{align*}
		\dot{\bs{x}} = \begin{bmatrix}
			p&1&\cdots&0&0\\
			0&p&\cdots&0&0\\
			\vdots&\vdots&\ddots&\vdots&\vdots\\
			0&0&\cdots&p&1\\
			0&0&\cdots&0&p
		\end{bmatrix}\bs{x} + \begin{bmatrix}
		0\\0\\\vdots\\0\\1
		\end{bmatrix} u &&\bs{y} = \begin{bmatrix}
		b&0&\cdots&0&0
		\end{bmatrix}\bs{x}
	\end{align*}
	
\end{enumerate}

	\tcblower
	Zum erstellen der Modalform:
	\begin{align}
		\text{Eigenwerte } \lambda_i \text{ von } \bs{F} berechnen\\
		\text{Eigenvektoren von } \bs{F} \text{ bilden } \bs{T}
	\end{align}
		\begin{tcolorbox}[colback=white!10!white,colframe=blue!70!black,title=Umrechnungsvorschrift in Modalform]
			\begin{align*}
			\bs{A} = \text{diag}\{\lambda_1,\dots,\lambda_n\} &
			 \\
			\bs{B} = \bs{T}^{-1}\bs{G} 
			&
			\hspace{1cm}\bs{C} = \bs{HT}
			\end{align*}
		\end{tcolorbox}

	\begin{tcolorbox}[colback=white!10!white,colframe=gray!70!black,title=Aussagen]
		Für $\bs{B}$ - wenn $b_i = 0$ $\Rightarrow x_i$ \textbf{nicht steuerbar}
		
		Für $\bs{C}$ - wenn $c_i = 0$ $\Rightarrow x_i$ \textbf{nicht beobachtbar}
		
		
		
	\end{tcolorbox}		
		
	
\end{tcolorbox}

\paragraph{Aufgabe:}
Stellen Sie folgende Übertragungsfunktion in Jordannormalform dar:
\begin{align*}
	G(s) &=  \frac{2s+4}{s^2(s^2+2s+4)} = \frac{1}{s^2} - \frac{1}{s^2+2s+4}
\end{align*}
\textbf{Lösung:}
\begin{align*}
	\begin{bmatrix}
	\dot{x}_1\\\dot{x}_2\\\dot{x}_3\\\dot{x}_4
		\end{bmatrix} = \left[\begin{array}{cc|cc}
		0&1&0&0\\
		0&0&0&0\\\hline
		0&0&0&1\\
		0&0&-4&-2\\
		\end{array}\right]\begin{bmatrix}
		x_1\\x_2\\x_3\\x_4
		\end{bmatrix} + \begin{bmatrix}
			0\\1\\0\\1
		\end{bmatrix}u && y = \begin{bmatrix}
		1 &0&-1&0
		\end{bmatrix}x
\end{align*}