\setcounter{section}{6}
\section{Bodediagramm}
\begin{tcolorbox}[colback=white!10!white,colframe=blue!50!white,title=Konstruktionsregeln]
    \textbf{Amplitudengang:}
    \begin{align*}
        &G(s) = c\cdot s^r\cdot \underbrace{G_1(s) \cdot G_2(s)\cdot \ldots \cdot G_k(s)}_{G^*(s)}\cdot e^{-T_ts}\\
        &G_i(s)  = (s+n_i)  \text{ reelle Nullstelle}\\
    \end{align*}
    \begin{enumerate}
        \item Eckfrequenzen aufsteigend sortieren
        \item niedrigste Frequenz ist die Startfrequenz $\omega_1$
        \item Startamplitude:\\ $A_{\text{dB}}(\omega_{\text{min}}) = 20 \log_{10}(|c\cdot G^{*}(0)|\cdot \omega_{\text{min}}^r)$
        \item Asymptote links vom Startpunkt:\\
        $r\cdot20 \frac{\text{dB}}{\text{Dekade}}$
        \item Wenn zwischen benachbarten Eckfreuqenzen 1 Dekade liegt $\Rightarrow$ Absenkung(POL)/Erhöhung(NST) um  $n*3$dB über dem jeweiligen Knickpunkt
        \item Jede Eckfrequenz verändert die Steigung folgendermaßen
                \begin{description}[leftmargin=14em,style=nextline]
                    \item[$G_i = (s+n_i)$] $+20\frac{\text{dB}}{\text{Dekade}}$
                    \item[$G_i = \frac{1}{(s+p_i)}$] $-20\frac{\text{dB}}{\text{Dekade}}$
                    \item[$G_i = (s^2 + 2D_im_is+m_i^2)$] $+40\frac{\text{dB}}{\text{Dekade}}$
                    \item[$G_i = \frac{1}{(s^2 + 2D_ip_is+p_i^2)}$] $-40\frac{\text{dB}}{\text{Dekade}}$
                \end{description}
        \item Jeder Term der Form $s^2 + 2D_im_is+m_i^2$ liefert einen Peak  am entsprechenden $\omega_i$. Eine Nullstelle unterhalb der Asymptote - eine Polstelle oberhalb der Asymptote.\\
        $(\pm)20\log(\frac{1}{2*D_i})$
    \end{enumerate}
    Exakter Wert: $A_{dB}(\omega) = 20 \cdot \log_{10} \left(G(\omega)\right)$
    \tcblower
    \textbf{Phasengang:}
    \begin{enumerate}
        \item Bestimme Startfrequenz\\
        $\varphi(0) = \begin{cases}
    r\cdot 90^\circ &\mbox{wenn } n \equiv c\cdot G^*(0)>0\\
    -180^\circ+r\cdot 90^\circ & \mbox{wenn } n \equiv c\cdot G^*(0)<0
        \end{cases} $
        \item an jeder Eckfrequenz springt die Phase:
        \begin{description}[leftmargin=14em,style=nextline]
            \item[$G_i = (s+n_i)$] $+90^\circ\cdot\mbox{sign}(n_i)$
            \item[$G_i = \frac{1}{(s+p_i)}$] $-90^\circ\cdot\mbox{sign}(p_i)$
            \item[$G_i = (s^2 + 2D_im_is+m_i^2)$] $+180^\circ\cdot\mbox{sign}(D_im_i)$
            \item[$G_i = \frac{1}{(s^2 + 2D_iq_is+q_i^2)}$] $-180^\circ\cdot\mbox{sign}(D_iq_i)$
        \end{description}
        \item Wenn $2D_iq_is = 0 \rightarrow$ ungedämpftes Polpaar $\rightarrow\mbox{sign}(0) = \pm 1$ (kann gewählt werden)
    \end{enumerate}
    Exakter Wert: \\
        $\varphi(\omega) = \sum\limits_{j} \arctan\left(-\frac{b}{a}\right)
        : \forall G_j(i\omega) \in G^* : \begin{cases}
            b + ai \\
            \frac{1}{a + bi}
        \end{cases}$
\end{tcolorbox}

\begin{tcolorbox}[colback=white!10!white,colframe=green!30!black,title=Phasenminimumsystem]
\textbf{Phasenminimumsystem} ist dann gegeben, wenn die Übertragungsfunktion
\begin{itemize}
    \item keinen Totzeitfaktor enthält
    \item G(s) hat Pole und Nullstelle nur in LHE
\end{itemize}
\end{tcolorbox}
