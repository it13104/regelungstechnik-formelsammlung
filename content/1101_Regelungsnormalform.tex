\begin{tcolorbox}[colback=white!10!white,colframe=blue!70!black,title=KOCHREZEPT: Regelungsnormalform]
    \textbf{Definition:}
    \begin{align}
        Y(s) &= \frac{b(s)}{a(s)} U(s) \\ & = \frac{b_1s^{n-1}+\cdots+b_{n-1}s+b_n}{s^n+a_1s^{n-1}+\cdots + a_{n-1}s+a_n} U(s)\label{eq:uebertragung}\\
        \dot{\bs{x}} &= 
        \begin{bmatrix}
            0 & 1 & \cdots & 0\\
            \vdots & \vdots & \ddots & \vdots \\
            0&0&\cdots&1\\
            -a_n&-a_{n-1}&\cdots&-a_1
        \end{bmatrix}\bs{x}+ 
        \begin{bmatrix}
            0 \\ \vdots \\0\\1
        \end{bmatrix} u \\
        y&= \begin{bmatrix}
            b_n & b_{n-1} & \cdots & b_1
        \end{bmatrix} \bs{x}
    \end{align}


    \tcblower
    Bilde die Inverse $\bs{\mathcal{C}}^{-1}$ von:
    \begin{align*}
        \bs{\mathcal{C}} = \begin{bmatrix}
            G & FG & \cdots & F^{n-1}G
        \end{bmatrix}
    \end{align*}
    Nehme letzte Zeile $\bs{t}_1^T$ von $\bs{\mathcal{C}}^{-1}$ 
    \begin{align*}
        \begin{bmatrix}
            0 & 0 & \cdots & 1
        \end{bmatrix} \bs{\mathcal{C}}^{-1}
    \end{align*}
    Stelle inverse der Transformationsmatrix $\bs{T}$ auf:
    \begin{align*}
        \bs{T}^{-1} &= \left[\begin{array}{l}
        \bs{t}_1^T \\
        \bs{t}_1^T \bs{F}\\
        \bs{t}_1^T \bs{F}^2\\
        \vdots\\
        \bs{t}_1^T \bs{F}^{n-1}
        \end{array}\right]
    \end{align*}
    
    \begin{tcolorbox}[colback=white!10!white,colframe=blue!70!black,title=Umrechnungsvorschrift Zustandsform]
        \begin{align*}
            \bs{B} = \begin{bmatrix}
            0\\ 0\\ \vdots \\1
            \end{bmatrix} \hspace{0.5cm}
            &
            \bs{A} = \bs{T}^{-1}\bs{FT} &
            \bs{C} = \bs{HT}
        \end{align*}
    \end{tcolorbox}
    
    \begin{tcolorbox}[colback=white!10!white,colframe=gray!70!black,title=Steuerbarkeit]
        Das System heißt \textbf{steuerbar}, wenn $\bs{\mathcal{C}}$ \textbf{nicht singulär} ist. (Es existiert eine Inverse von $\bs{\mathcal{C}}$)
    \end{tcolorbox}
\end{tcolorbox}

