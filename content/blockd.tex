\section{Blockdiagramme}
\begin{tcolorbox}[colback=white!10!white,colframe=blue!50!black,title=Regeln]
	
	\begin{figure}[H]
		\begin{subfigure}{.3\textwidth}
			\centering
			\includegraphics[width=1\textwidth]{content/img/kettenschaltung}
			\label{fig:kettenschaltung}
			\caption{Kettenschaltung}
		\end{subfigure}%
		\begin{subfigure}{.3\textwidth}
	\centering
	\includegraphics[width=1\textwidth]{content/img/parallel}
	\caption{Parallelschaltung}
	\label{fig:parallel}
		\end{subfigure}%
			\begin{subfigure}{.3\textwidth}
				\centering
				\includegraphics[width=1\textwidth]{content/img/back}
				\caption{Rückführung}
				\label{fig:back}
			\end{subfigure}%
	\end{figure}
\begin{figure}[H]

			\begin{subfigure}{.5\textwidth}
			\centering
			\includegraphics[width=1\textwidth]{content/img/verschiebung_punkt}
			\caption{Pickoff-Verschiebung}
			\label{fig:pickoff}
			\end{subfigure}%
			\begin{subfigure}{.5\textwidth}
				\centering
				\includegraphics[width=1\textwidth]{content/img/sum_front}
				\caption{Summationspukt-Verschiebung}
				\label{fig:sum}
			\end{subfigure}%
			

\end{figure}
\begin{figure}[H]
	\begin{subfigure}{.5\textwidth}
		\centering
		\includegraphics[width=1\textwidth]{content/img/satz}
		\caption{}
		\label{fig:satz}
	\end{subfigure}%
\end{figure}
\end{tcolorbox}

\begin{tcolorbox}[colback=white!10!white,colframe=gray!50!black,title=Beispiele]
	\begin{figure}[H]
		
		\begin{minipage}{.5\textwidth}
			\centering
			\includegraphics[width=1\textwidth]{content/img/example_1}
	
		\end{minipage}%
		\begin{minipage}{.5\textwidth}
			\begin{align*}
				T(s) = \frac{\frac{2s+4}{s^2}}{1+\frac{2s+4}{s^2}}
			\end{align*}
		\end{minipage}%
		
		
	\end{figure}
	\begin{figure}[H]
	\noindent\rule[0.2ex]{\linewidth}{1pt}
	
		\begin{minipage}{.5\textwidth}
			\centering
			\includegraphics[width=1\textwidth]{content/img/example_2}
			
		\end{minipage}%
		\begin{minipage}{.5\textwidth}
			\begin{align*}
			&T(s) = \frac{\frac{G_1G_2}{1-G_1G_3}}{1+\frac{G_1G_2G_4}{1-G_1G_3}}\cdot\\& \cdot(G_5+\frac{G_6}{G_2})\\
			&= \frac{G_1G_2G_5+G_1G_6}{1-G_1G_3+G_1G_2G_4}
			\end{align*}
		\end{minipage}%
		
		
	\end{figure}

\end{tcolorbox}