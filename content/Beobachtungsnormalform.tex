\begin{tcolorbox}[colback=white!10!white,colframe=blue!70!black,title=KOCHREZEPT: Beobachtungsnormalform]
	Partialbruchzerlegung der Übertragungsfunktion $\Rightarrow$ Modus
\\

\begin{enumerate}
	\item Alle Pole sind \underline{reell} und \underline{einfach}:
	\begin{align*}
	G(s) = \frac{b_1}{s-p_1}+ \cdots +\frac{b_n}{s-p_n}
	\end{align*}
	\begin{align*}
	\dot{\bs{x}} = 
	\begin{bmatrix}[1]
	p_1 & 0 &\cdots & 0 \\
	0&p_2 &\cdots & 0  \\
	\vdots&\vdots&\ddots&\vdots\\
	0&0&\cdots&p_n
	\end{bmatrix}\bs{x}+ 
	\begin{bmatrix}
	1\\ 1 \\ \vdots \\1
	\end{bmatrix} u 	
	\end{align*}
	\begin{align*}
	y= \begin{bmatrix}
	b_1 & b_2&\cdots &b_n
	\end{bmatrix} \bs{x}
	\end{align*}
	\textbf{Jordanform -  wenn die Nullstellen nicht reell sind:}
	\item Komplexes Polpaar $\Rightarrow$ Partialbruch der Art.
	Blockmatrizen für das Polpaar werden an entsprechender Stelle eingefügt
	\begin{align*}
		\frac{b_{i+1}s+b_i}{s^2+a_is+a_{i+1}}
	\end{align*}
	\begin{align*}
		\dot{\bs{x}} = \begin{bmatrix}
			0 & 1\\
			-a_{i+1} & -a_i
		\end{bmatrix}\bs{x} + \begin{bmatrix}
		0\\1
		\end{bmatrix}u & & \bs{y} = \begin{bmatrix}
		b_i & b_{i+1}
		\end{bmatrix}\bs{x}
	\end{align*}
	
	\item Ein $m$-facher Pol $\frac{b}{(s-p)^m}$
	\begin{align*}
		\dot{\bs{x}} = \begin{bmatrix}
			p&1&\cdots&0&0\\
			0&p&\cdots&0&0\\
			\vdots&\vdots&\ddots&\vdots&\vdots\\
			0&0&\cdots&p&1\\
			0&0&\cdots&0&p
		\end{bmatrix}\bs{x} + \begin{bmatrix}
		0\\0\\\vdots\\0\\1
		\end{bmatrix} u &&\bs{y} = \begin{bmatrix}
		b&0&\cdots&0&0
		\end{bmatrix}\bs{x}
	\end{align*}
	
\end{enumerate}

	\tcblower
	Bilde die Inverse $\bs{\mathcal{O}}^{-1}$ und nehme letzte \textbf{Spalte} $\bs{t}_1$ von $\bs{\mathcal{O}}^{-1}$
		
		\begin{align*}
		\bs{\mathcal{O}} &= 
		\left[\begin{array}{l}
		\bs{H}\\
		\bs{HF}\\
		\vdots\\
		\bs{HF}^{n-1}
		\end{array}\right] \rightarrow \bs{\mathcal{O}}^{-1} & 	 \bs{t}_1 = \bs{\mathcal{O}}^{-1} 	\begin{bmatrix}
		0 \\ 0 \\ \vdots \\ 1
		\end{bmatrix}
		\end{align*}

	Die Transformationsmatrix bilden $\bs{T}$ nach Vorschrift:
	\begin{align*}
	\bs{T} &= \left[\begin{array}{llll}
	\bs{t}_1 & \bs{Ft}_1 & \dots & \bs{F}^{n-1}\bs{t}_1
	\end{array}\right]
	\end{align*}
	
	\begin{tcolorbox}[colback=white!10!white,colframe=blue!70!black,title=Umrechnungsvorschrift in Beobachtungsnormalform]
		\begin{align*}
		\bs{C} = \begin{bmatrix}
		0 & 0& \dots &1
		\end{bmatrix} 
		\\
		 \bs{A} = \bs{T}^{-1}\bs{FT} 
		\\
	    	\bs{B} = \bs{T}^{-1}\bs{G}
		\end{align*}
	\end{tcolorbox}
	
	\begin{tcolorbox}[colback=white!10!white,colframe=gray!70!black,title=Beobachtbarkeit]
		Das System heißt \textbf{beobachtbar}, wenn $\bs{\mathcal{O}}$ \textbf{nicht singulär} ist. (Es existiert eine Inverse von $\bs{\mathcal{O}}$)
	\end{tcolorbox}
\end{tcolorbox}