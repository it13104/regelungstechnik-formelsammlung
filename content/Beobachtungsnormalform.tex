\begin{tcolorbox}[colback=white!10!white,colframe=blue!70!black,title=KOCHREZEPT: Beobachtungsnormalform]
    \textbf{Definition:}
Aus der Funktion \ref{eq:uebertragung} stellt man die Matrizen auf:
	\begin{align*}
		\dot{\bs{x}} &= 
		\begin{bmatrix}
			0 &  \cdots & 0 &-a_n\\
			1 &\cdots & 0 & -a_{n-1} \\
			\vdots&\ddots&\vdots&\vdots\\
			0&\cdots&1&-a_1
		\end{bmatrix}\bs{x}+ 
		\begin{bmatrix}
			b_n \\ b_{n-1} \\ \vdots \\b_1
		\end{bmatrix} u \\
		y&= \begin{bmatrix}
			0 & \cdots & 0& 1
		\end{bmatrix} \bs{x}
	\end{align*}
	\begin{tcolorbox}[colback=white!10!white,colframe=gray!70!black,title=Umrechnungsregeln]
		\begin{align*}
			\bs{A}_O = \bs{A}_C^T &\hspace{0.3cm} \bs{B}_O = \bs{C}_C^T \\
			\bs{C}_O = \bs{B}_C^T   &\hspace{0.3cm} D_O = D_C
		\end{align*}
	
	\end{tcolorbox}
    \tcblower
    Bilde die Inverse $\bs{\mathcal{O}}^{-1}$ und nehme letzte \textbf{Spalte} $\bs{t}_1$ von $\bs{\mathcal{O}}^{-1}$
        
        \begin{align*}
        \bs{\mathcal{O}} &= 
        \left[\begin{array}{l}
        \bs{H}\\
        \bs{HF}\\
        \vdots\\
        \bs{HF}^{n-1}
        \end{array}\right] \rightarrow \bs{\mathcal{O}}^{-1} &      \bs{t}_1 = \bs{\mathcal{O}}^{-1}     \begin{bmatrix}
        0 \\ 0 \\ \vdots \\ 1
        \end{bmatrix}
        \end{align*}

    Die Transformationsmatrix bilden $\bs{T}$ nach Vorschrift:
    \begin{align*}
    \bs{T} &= \left[\begin{array}{llll}
    \bs{t}_1 & \bs{Ft}_1 & \dots & \bs{F}^{n-1}\bs{t}_1
    \end{array}\right]
    \end{align*}
    
    \begin{tcolorbox}[colback=white!10!white,colframe=blue!70!black,title=Umrechnungsvorschrift in Beobachtungsnormalform]
        \begin{align*}
        \bs{C} = \begin{bmatrix}
        0 & 0& \dots &1
        \end{bmatrix} 
        \\
         \bs{A} = \bs{T}^{-1}\bs{FT} 
        \\
            \bs{B} = \bs{T}^{-1}\bs{G}
        \end{align*}
    \end{tcolorbox}
    
    \begin{tcolorbox}[colback=white!10!white,colframe=gray!70!black,title=Beobachtbarkeit]
        Das System heißt \textbf{beobachtbar}, wenn $\bs{\mathcal{O}}$ \textbf{nicht singulär} ist. (Es existiert eine Inverse von $\bs{\mathcal{O}}$)
    \end{tcolorbox}
\end{tcolorbox}
