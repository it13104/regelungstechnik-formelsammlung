\section{Diskrete Systeme}

    \begin{tcolorbox}[colback=white!10!white,colframe=green!30!black,title=Zusammenhang $s$ und $z$]

\textbf{Abbildung:}    

Linke komplexe s-Halbebene $\Longrightarrow$ Einheitskreis $|z|=1$

$z = e^{sT}$ ist die Abbildungsvorschrift
\end{tcolorbox}        
    \begin{tcolorbox}[colback=white!10!white,colframe=green!30!black,title=$z$-Übertragungsfunktion]
        \begin{align*}
            \mathcal{Z}\{ f(k-n)\} = z^{-n}F(z)
        \end{align*}
        Allgemeine DGL (2. Ordnung):
        \begin{align*}
        &y(k)    = -a_1y(k-1)-a_2(k-2)+b_0u(k)\\&+b_1u(k-1)+b_2u(k-2)\\
        &\mathcal{Z}\{ y(k)\}  = Y(z)\\
        &Y(z) = (-a_1z^{-1}-a_2z^{-2})Y(z) 
        +(b_0+b_1z^{-1}+b_2z^{-2})U(z)\\
        &\frac{Y(z)}{U(z)} = \frac{b_0+b_1z^{-1}+b_2z^{-2}}{1+a_1z^{-1}+a_2z^{-2}}
        \end{align*}
        \textbf{Ablesen des Pols:}
            \begin{align*}
            & Y(z) = \frac{1}{1-\alpha z^-1} &\Rightarrow z_1 = \alpha
            \end{align*}
        \begin{tcolorbox}[colback=white!10!white,colframe=green!30!black]
        \textbf{Endwertsatz der $z$-Transformation:}
        
        Kontinuierliches System:
        \begin{align*}
        \lim\limits_{t\to\infty} x(t) = x_{\text{stat}} =     \lim\limits_{s\to 0 }sX(s)
        \end{align*}
    Diskretes System:
        \begin{align*}
             \lim\limits_{k\to\infty} x(k) =  x_{\text{stat}} = \lim\limits_{z\to 1 }X(z)
        \end{align*}
        \end{tcolorbox}    
    \end{tcolorbox}    
    \begin{tcolorbox}[colback=white!10!white,colframe=blue!30!black,title=$z$-Übertragungsfunktion: Halteglied und kontinuierliches System in Reihe] 
        Das ist eine Approximation über Reihenschaltung Halteglied und kontinuierliches System:
        \begin{align*}
            G(z) &= \frac{z-1}{z}\mathcal{Z}\left\{\frac{G(s)}{s}\right\}
        \end{align*}
        


    \end{tcolorbox}            
\subsection{Diskrete Fundamentalmatrix}
    \begin{tcolorbox}[colback=white!10!white,colframe=green!30!black,title=Definition] 
        \begin{align*}
            \bs{A}_d(T)= e^{\bs{A}T} && B_d = \int_{0}^{T}\bs{A}_d
        (\tau)d\tau    \cdot \bs{B} 
\end{align*}
Mit Taylorzerlegung kann man $A_d$ berechnen:
\begin{align*}
    e^{\bs{A}t}  &= \bs{I} + \bs{A}t+\bs{A}^2\frac{t^2}{2!} + \bs{A}^3\frac{t^3}{3!}+ \ldots\\
    \frac{d}{dt}e^{\bs{A}t} &= \bs{A} + \bs{A}^2t+\bs{A}^3\frac{t^2}{2!} + \ldots
\end{align*}
    \end{tcolorbox}    
        \begin{tcolorbox}[colback=white!10!white,colframe=blue!30!black,title=KOCHREZEPT: Fundamentalmatrix Berechnung] 
            \begin{align*}
            \bs{A}_d(T) &= \mathcal{L}^{-1}\{(s\bs{I}-\bs{A})^{-1}\}\\
            \bs{B}_d(T) &= \mathcal{L}^{-1}\{(\frac{1}{s}\bs{A}_d(s)\cdot \bs{B})\}
            \end{align*}
            Für $2\times2$-Matrix lässt sich $A_d$ so bestimmen:
            \begin{align*}
                \bs{A}_d(s) &= \frac{1}{|s\bs{I}-\bs{A}|} adj(s\bs{I}-\bs{A})\\
                \bs{A}_d(T) &= \mathcal{L}^{-1}\{A_d(s)\}
            \end{align*}
            \tcblower
            \begin{align*}
                \mathcal{C}_d &= \begin{bmatrix}
                \bs{B}_d & \bs{A}_d\bs{B}_d & \ldots & \bs{A}_d^{n-1}\bs{B}_d
                \end{bmatrix}\\
                \mathcal{O}_d &= \begin{bmatrix}
                \bs{C} \\\bs{C}\bs{A}_d \\ \vdots \\ \bs{C}\bs{A}_d^{n-1}
                \end{bmatrix}
            \end{align*}
            \begin{tcolorbox}[colback=white!10!white,colframe=gray!70!black,title=Steuerbarkeit und Beobachtbarkeit]
                Das zeitdiskrete System ($\bs{A}_d,\bs{B}_d$) ist vollständig \textbf{steuerbar} bzw. \textbf{beobachtbar}, wenn  kontinuierliches System ($\bs{A},\bs{B}$)  es ist und wenn die Abtastzeit folgendes erfüllt:
                 
                \begin{align*}
                    T < \frac{\pi}{\omega_{j,\text{max}}}
                \end{align*}
                
                $\omega_{j,\text{max}}$ ist betraglich größter Eigenwert von $A$
            \end{tcolorbox}    
        \end{tcolorbox}

