\setcounter{section}{16}
\section{Diskrete Systeme}
\begin{tcolorbox}[colback=white!10!white,
                  colframe=green!30!black,
                  title=Zusammenhang $s$ und $z$]
\textbf{Abbildung:}

Linke komplexe s-Halbebene $\Longrightarrow$ Einheitskreis $|z|=1$

$z = e^{sT}$ ist die Abbildungsvorschrift
\end{tcolorbox}

\begin{tcolorbox}[colback=white!10!white,
                  colframe=green!30!black,
                  title=$z$-Übertragungsfunktion]
    \begin{align*}
        \mathcal{Z}\{ f(k-n)\} = z^{-n}F(z)
    \end{align*}
    Allgemeine DGL (2. Ordnung):
    \begin{align*}
        &y(k)    = -a_1y(k-1)-a_2(k-2)+b_0u(k)\\&+b_1u(k-1)+b_2u(k-2)\\
        &\mathcal{Z}\{ y(k)\}  = Y(z)\\
        &Y(z) = (-a_1z^{-1}-a_2z^{-2})Y(z) 
        +(b_0+b_1z^{-1}+b_2z^{-2})U(z)\\
        &\frac{Y(z)}{U(z)} = \frac{b_0+b_1z^{-1}+b_2z^{-2}}{1+a_1z^{-1}+a_2z^{-2}}
    \end{align*}
    \textbf{Ablesen des Pols:}
    \begin{align*}
        & Y(z) = \frac{1}{1-\alpha z^-1} &\Rightarrow z_1 = \alpha
    \end{align*}
    \begin{tcolorbox}[colback=white!10!white,colframe=green!30!black]
        \textbf{Endwertsatz der $z$-Transformation:}
        
        Kontinuierliches System:
        \begin{align*}
            \lim\limits_{t\to\infty} x(t) = x_{\text{stat}} = \lim\limits_{s\to 0 }sX(s)
        \end{align*}
        Diskretes System:
        \begin{align*}
             \lim\limits_{k\to\infty} x(k) =  x_{\text{stat}} = \lim\limits_{z\to 1 }X(z)
        \end{align*}
    \end{tcolorbox}
\end{tcolorbox}

\begin{tcolorbox}[colback=white!10!white,
                  colframe=blue!30!black,
                  title=$z$-Übertragungsfunktion: Halteglied und kontinuierliches System in Reihe]
    Das ist eine Approximation über Reihenschaltung Halteglied und kontinuierliches System:
    \begin{align*}
        G(z) &= \frac{z-1}{z}\mathcal{Z}\left\{\frac{G(s)}{s}\right\}
    \end{align*}
\end{tcolorbox}

\begin{tcolorbox}[colback=white!10!white,
                  colframe=blue!30!black,
                  title=Reglerentwurf - diskrete Äquivalente]
    \textbf{Vorgehen:}
    \begin{enumerate}
        \item Entwerfe kontinuierlichen Regler
        \item Diskretisiere den Regler
        \item Bestätige Entwurf durch diskrete Analyse
    \end{enumerate}
    Genügt eine lineare Interpolation kann die \textbf{Trapezregel} 
    (Substitution) zur \textbf{Diskretisierung} verwendet werden:
    \begin{align*}
        s &= \frac{2}{T} \cdot \frac{1-z^{-1}}{1+z^{-1}}
    \end{align*}
\end{tcolorbox}