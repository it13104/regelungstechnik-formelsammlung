\section{Beobachterentwurf}

\begin{tcolorbox}[colback=white!10!white,colframe=green!30!black] 
	Manche Zustandsgrößen können nur geschätzt werden (Messung nicht möglich). Man nutzt einen Schätzwert (Simulation) $\hat{x}$. Mit dem Beobachterentwurf wird dafür gesorgt, dass der Schätzfehler auf Null abklingt.
		\textbf{Vorgehensweise}
		\begin{enumerate}
			\item Wunschpole zu Wunsch-Charakteristischen Gleichung umformen
			\item Bilde $\det(s\bs{I}-(\bs{F}-\bs{LH}))\Rightarrow$ - Charakteristische Gleichung abhängig von $\bs{L} = \begin{bmatrix}
			l_1&\ldots&l_n
			\end{bmatrix}^T$ 
			\item Koeffizientenvergleich und Werte von $\bs{L}$ bestimmen
		\end{enumerate}
	\tcblower
\textbf{System in Beobachtungnormalform}
Die Charakteristische Gleichung lässt sich einfach ablesen aus
\begin{align*}
	&\bs{F}-\bs{LH} = 
	\begin{bmatrix}
	0 &  \cdots & 0 &-a_n-l_1\\
	1 &\cdots & 0 & -a_{n-1}-l_2 \\
	\vdots&\ddots&\vdots&\vdots\\
	0&\cdots&1&-a_1-l_n
	\end{bmatrix}	\\
	&s^n +(a_1+l_n)s^(n-1)+ (a_2+l_{n-1}+\ldots+(a_n+l_1))=0
\end{align*}


\end{tcolorbox}