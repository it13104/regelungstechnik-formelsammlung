\section{Regler-Entwurf}
\begin{tcolorbox}[colback=white!10!white,colframe=green!30!black] 

    \begin{enumerate}
        \item     Die Zustandsrückführung ist : $u = -\bs{Kx}$ einsetzen in $\dot{\bs{x}} = \bs{Fx}+\bs{G}u$
        \item $\dot{\bs{x}} = (\bs{F}-\bs{GK})\bs{x}$
        \item Für vorgegebene Pole $s_i$ charakteristische Gleichung aufstellen
        \item $\det (s\bs{I}-\bs{F}+\bs{GK})$ ergibt die charakteristische Gleichung in Abhängigkeit von $\bs{K} = \begin{bmatrix}
        k_1 & \ldots & kn
        \end{bmatrix}$
        \item Koeffizentenvergleich zwischen beiden charakteristischen Gleichungen $\Rightarrow \bs{K}$ 
    \end{enumerate} 
\tcblower
\textbf{System in Regelungsnormalform}
$K$ ist besonders einfach, wenn ein System in RNF vorliegt.
\begin{align*}
    \bs{A}_C-\bs{B}_C\bs{K} = \begin{bmatrix}
    0 & 1& 0&\ldots&0\\     0 & 0& 1&\ldots&0\\
    0 & 0& 0&\ddots&0\\
    0 & 0& 0&\ldots&1\\
    -a_n-k_1 & -a_{n-1}-k_2& &\ldots&-a_{1}-k_n\\
    \end{bmatrix}
\end{align*}
Die charakteristische Gleichung ergibt sich sofort zu:
\begin{align*}
    s^n +(a_1+k_n)s^{n-1}+(a_2+k_{n-1})s^{n-2}+\ldots+(a_n+k_1) =0 
\end{align*}

Man kann direkt den Koeffizientenvergleich machen - Determinantenbildung entfällt.
    
\end{tcolorbox}

\begin{tcolorbox}[colback=white!10!white,colframe=green!30!black,title=Integralregelung]
    Charakteristische Gleichung $0=det(s\bs{I}-\bs{A}_a+\bs{B}_a*\bs{K})$
    Integralregelung  sorgt dafür der Fehler bei Änderung der Streckenparameter Null wird. Das System wir um einen Integralen Teil erweitert.
    \begin{align*}
        \begin{bmatrix}
        \dot{    \bs{x}_I} \\ \dot{    \bs{x}}
        \end{bmatrix} &= \begin{bmatrix}
        0 & \bs{H} \\ 0 & \bs{F}
        \end{bmatrix}\begin{bmatrix}
            \bs{x}_I \\    \bs{x}
        \end{bmatrix}+
        \begin{bmatrix}
        0 \\ \bs{G}
        \end{bmatrix} u - \begin{bmatrix}
        1 \\ 0
        \end{bmatrix}r
    \end{align*}
    mit dem Regler:
    \begin{align*}
        u = -\begin{bmatrix}
         K_1 & \bs{K_0}
        \end{bmatrix} \begin{bmatrix}
            \bs{x}_I \\\bs{x}
        \end{bmatrix} = - \bs{K} \begin{bmatrix}
        \bs{x}_I \\\bs{x}
        \end{bmatrix}
    \end{align*}
    
    \begin{figure}[H]
\centering
\includegraphics[width=0.7\linewidth]{content/img/integral}

\end{figure}
\tcblower
\textbf{Beispiel:}
Übertragungsfunktion $\frac{Y(s)}{U(s)} = \frac{1}{s+3}$\\
Systembeschreibung $F = -3, G = 1, H = 1$

Gewünscht Pole bei $s=-5$ + Integralanteil \\$\Rightarrow a_C(s) = s^2+10s+25$

Die erweiterte Systembeschreibung ergibt sich zu:
\begin{align*}
        \begin{bmatrix}
        \dot{    \bs{x}_I} \\ \dot{    \bs{x}}
        \end{bmatrix} = \begin{bmatrix}
            0 & 1\\ 0 & -3
        \end{bmatrix} \begin{bmatrix}
        \bs{x}_I \\\bs{x}
        \end{bmatrix} + \begin{bmatrix}
        0 \\1
        \end{bmatrix}u - \begin{bmatrix}
        1\\0
        \end{bmatrix}r
\end{align*}
Die Rückführmatrix 
\begin{align*}
    &\det\left( s\bs{I} - \begin{bmatrix}
    0 & 1\\ 0 & -3
    \end{bmatrix}+ \begin{bmatrix}
    0 \\1
    \end{bmatrix}\bs{K}\right) \overset{!}{=} s^2 +10s +25\\
    &\bs{K} = \begin{bmatrix}
        K_1 & K_2
    \end{bmatrix} = \begin{bmatrix}
    25 & 7
    \end{bmatrix}
\end{align*}
\end{tcolorbox}
